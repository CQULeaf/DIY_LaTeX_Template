% template.tex

% 引入必要的包
\usepackage{amsmath, amssymb, amsfonts, amsthm}  % AMS 提供的数学排版增强工具,支持公式、符号、定理环境等
\usepackage{graphicx} % 插入图片
\usepackage{geometry} % 自定义页面边距
\usepackage{fancyhdr} % 自定义页眉页脚
\usepackage{titlesec} % 自定义章节标题格式
\usepackage{hyperref} % 生成超链接(目录、引用、网址等)
\usepackage{xcolor} % 支持彩色文字和图形
\usepackage{listings} % 排版代码块
\usepackage{booktabs} % 绘制高质量表格
\usepackage{caption} % 自定义图表标题样式
\usepackage{enumitem} % 自定义列表(itemize/enumerate)间距和符号
\usepackage{tcolorbox} % 用于创建带颜色、阴影的文本框
\usepackage{eso-pic} % 在页面底层添加元素
\usepackage{fontawesome5} % 使用 Font Awesome 图标(如 GitHub 图标)
\usepackage{tocloft} % 自定义目录(ToC)、图表目录样式
\usepackage{background} % 添加背景
\usepackage{varwidth} % 创建“可变宽度盒子”

% 文献引用
\usepackage[utf8]{inputenc}
\usepackage[backend=biber,style=numeric,citestyle=numeric]{biblatex}  
\addbibresource{references.bib}  % 引入 references.bib 文件
\hypersetup{colorlinks=true, linkcolor=black, urlcolor=black, citecolor=black}

% 设置图片目录
\graphicspath{{Figures/}}

% 页面布局设置
\geometry{left=2.5cm, right=2.5cm, top=2.5cm, bottom=2.5cm}
\setlength{\headheight}{14.5pt}
\addtolength{\topmargin}{-2.5pt}

% 修改目录的字体颜色为黑色
\hypersetup{
    colorlinks=true,
    linkcolor=black,  % 将目录超链接设置为黑色
    urlcolor=magenta,
    citecolor=black,
    pdfborder={0 0 0}
}

% 页眉页脚设置
\pagestyle{fancy}
\fancyhf{}
\fancyhead[L]{\leftmark}
\fancyhead[R]{\thepage}

\fancyfoot[R]{\href{https://github.com/CQULeaf}{\faGithub\ CQULeaf}}

% 代码高亮设置
\lstset{
    basicstyle=\ttfamily\footnotesize,
    keywordstyle=\color{blue},
    commentstyle=\color{gray},
    stringstyle=\color{red},
    frame=single,
    breaklines=true
}

% -- 加载 tcolorbox 高级库 --
\tcbuselibrary{breakable, theorems, skins}

% 定义颜色
\definecolor{TheoremColor}{RGB}{34,139,34}    % 深绿(定理)
\definecolor{DefColor}{RGB}{45,52,151}        % 深蓝(定义)
\definecolor{ExampleColor}{RGB}{226,135,67}   % 橙色(例子)
\definecolor{ProofColor}{RGB}{34,139,34}      % 证明用绿

% -- 定理环境(编号 1.1, 1.2...)--
\newtcbtheorem[number within=section]{theorem}{Theorem}{
  enhanced,
  frame empty, interior empty,
  colframe=TheoremColor!50!white,
  coltitle=TheoremColor!50!black,
  colbacktitle=TheoremColor!15!white,
  fonttitle=\bfseries,
  borderline={0.5mm}{0pt}{TheoremColor!15!white},
  borderline={0.5mm}{0pt}{TheoremColor!50!white,dashed},
  attach boxed title to top left={yshift=-2mm, xshift=2mm},
  boxed title style={boxrule=0.4pt},
  varwidth boxed title,
  breakable
}{thm}

% -- 定义环境 --
\newtcbtheorem[number within=section]{definition}{Definition}{
  enhanced,
  frame empty, interior empty,
  colframe=DefColor!50!white,
  coltitle=DefColor!50!black,
  colbacktitle=DefColor!15!white,
  fonttitle=\bfseries,
  borderline={0.5mm}{0pt}{DefColor!15!white},
  borderline={0.5mm}{0pt}{DefColor!50!white,dashed},
  attach boxed title to top left={yshift=-2mm, xshift=2mm},
  boxed title style={boxrule=0.4pt},
  varwidth boxed title,
  breakable
}{def}

% -- 例子环境 --
\newtcbtheorem[number within=section]{example}{Example}{
  enhanced,
  frame empty, interior empty,
  colframe=ExampleColor!50!white,
  coltitle=ExampleColor!50!black,
  colbacktitle=ExampleColor!15!white,
  fonttitle=\bfseries,
  borderline={0.5mm}{0pt}{ExampleColor!15!white},
  borderline={0.5mm}{0pt}{ExampleColor!50!white,dashed},
  attach boxed title to top left={yshift=-2mm, xshift=2mm},
  boxed title style={boxrule=0.4pt},
  varwidth boxed title,
  breakable
}{ex}

% -- 证明环境 --
\tcolorboxenvironment{proof}{
  blanker,
  breakable,
  left=5mm,
  before skip=10pt,
  after skip=10pt,
  borderline west={1mm}{0pt}{ProofColor!50!white}
}

% 添加背景
\backgroundsetup{scale=0.4, angle=0, opacity = 1,contents = {\includegraphics[width=\paperwidth, height=\paperwidth, keepaspectratio]{Figures/CQU-Logo.png}}}